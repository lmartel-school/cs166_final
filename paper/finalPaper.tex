\documentclass[12pt]{article}
\usepackage{amsmath}
\usepackage{amsthm}
\usepackage{amssymb}
\usepackage{enumerate}
\usepackage{graphicx}
%\usepackage{fullpage}
\usepackage[top=1in, bottom=1in, left=0.8in, right=1in]{geometry}
\usepackage{multicol}
\usepackage{wrapfig}
\usepackage{units}
\usepackage{setspace}
\doublespacing

\setlength{\columnsep}{0.1pc}

\title{Ropes - Alternative String Representation}
\author{Leo Martel, Paul Martinez, Andy Moreland}
\date{\today}
\begin{document}

\maketitle
\vspace{-0.3in}
\rule{\linewidth}{0.4pt}

%%%%%%%%%%%%%%%%%%%%%%%%%%%%%%%%%%%%%%%%%%%

\section{Introduction}

In this paper we will discuss the Rope data structure, a data structure intended to serve as a more robust and more performant alternative to the tradtional String type offered in most languages.
The seminal paper regarding Ropes was written by Hans-J. Boehm, Russ Atkinson and Michael Plass in 1995.
We will provide an overview of their paper, beginning with their justification for Ropes in the first place.
We will continue with a more technical recap of the implmentation details and running time guarantees of a rope.
As an exercise we have implemented our own Rope and we will give some rough benchmark estimates comparing various operations.
Finally, in order to more easily show off the Rope data structure, we will discuss a visualization we created that allows one to actively interact with a Rope data structure and visualize its internal structure.

\section{Justification for Expanded String Type}

Boehm, Russ and Plass (BRP) begin their paper by discussing some of the faults of traditional string types and the various ways that they could be improved.
We will provide a brief summary of their points.

Before attacking `the traditional string type,' we must first define it.
This traditional string type we refer to is the crude fixed length arrays of characters offered
by languages such as C as Pascal.
We can usually access individual characters via some sort of array access and perform higher level operations such as concatenation or substrings through library functions.
Such implementations usually do not include any metadata along with the array (such as its length), and are usually mutable.

BRP list immutable strings as one of the characteristics of a more robust string type, noting that strings are often used to communicate between modules and thus one should be able to operate on a string without risk of modifying the original owner's copy of the string.

\section{Implementation and Running Time}

Maybe splay trees don't work.

\section{Benchmarking}

We wrote some code.

\section{Visualization}

Text editors, wooo.

\end{document}
